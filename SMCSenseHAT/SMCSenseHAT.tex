%%%%%%%%%%%%%%%%%%%%%%%%%%%%%%%%%%%%%%%%%%%%%%%%%%%%%%%%%%%%%%%%%%%%%%%%%%%%%
%
%  System        : 
%  Module        : 
%  Object Name   : $RCSfile$
%  Revision      : $Revision$
%  Date          : $Date$
%  Author        : $Author$
%  Created By    : Robert Heller
%  Created       : Wed May 31 20:05:00 2017
%  Last Modified : <170531.2304>
%
%  Description 
%
%  Notes
%
%  History
% 
%%%%%%%%%%%%%%%%%%%%%%%%%%%%%%%%%%%%%%%%%%%%%%%%%%%%%%%%%%%%%%%%%%%%%%%%%%%%%
%
%    Copyright (C) 2017  Robert Heller D/B/A Deepwoods Software
%			51 Locke Hill Road
%			Wendell, MA 01379-9728
%
%    This program is free software; you can redistribute it and/or modify
%    it under the terms of the GNU General Public License as published by
%    the Free Software Foundation; either version 2 of the License, or
%    (at your option) any later version.
%
%    This program is distributed in the hope that it will be useful,
%    but WITHOUT ANY WARRANTY; without even the implied warranty of
%    MERCHANTABILITY or FITNESS FOR A PARTICULAR PURPOSE.  See the
%    GNU General Public License for more details.
%
%    You should have received a copy of the GNU General Public License
%    along with this program; if not, write to the Free Software
%    Foundation, Inc., 675 Mass Ave, Cambridge, MA 02139, USA.
%
% 
%
%%%%%%%%%%%%%%%%%%%%%%%%%%%%%%%%%%%%%%%%%%%%%%%%%%%%%%%%%%%%%%%%%%%%%%%%%%%%%

\chapter{SMCSenseHAT: Stall Motor Control and Sense HAT}

This is a circuit board to for an add-on board for a Raspberry Pi B+ that will
control  two  stall-motor  turnout  motors for a model  railroad.  It also has
sense  logic to return the state of the  turnouts,  using one pole of the DPDT
contacts in the stall-motor (typical of Tortoise stall-motors).

The circuit board uses a 40pin header socket to connect to the 40pin header on
the  Raspberry Pi B+ and can use a  stack-through  header to allow  additional
boards to be stacked on top of it.

\section{GPIO Pins Used and stacking restrictions.}

This board uses four GPIO pins:

\begin{description}
\item[WiringPi 0, BCM 17] Motor Select 1: select the position of stall motor 
1. 
\item[WiringPi 1, BCM 18] Motor Select 2: select the position of stall motor 
2. 
\item[WiringPi 2, BCM 27] Point Sense 1: return the state of the points for 
stall motor 1. 
\item[WiringPi 3, BCM 22] Point Sense 2: return the state of the points for 
stall motor 2. 
\end{description}

Each of the motor drive circuits is through a transistor that can handle 1 amp 
continuious colector current, which is way more needed to drive typical stall 
motor.  It is enough to drive a pair of stall motors, wired in parallel as 
would be the case for a cross over.

Because this board is hardwired to use four specicic GPIO pins it is not 
possible to use two or more of these boards on given Raspberry Pi.  But in the 
SMCSenseHAT1 directory is a nearly identical board, that uses a different set 
of four GPIO pins:

\begin{description}
\item[WiringPi 4, BCM 23] Motor Select 1: select the position of stall motor 
1. 
\item[WiringPi 5, BCM 24] Motor Select 2: select the position of stall motor 
2. 
\item[WiringPi 6, BCM 25] Point Sense 1: return the state of the points for 
stall motor 1. 
\item[WiringPi 7, BCM 4] Point Sense 2: return the state of the points for 
stall motor 2. 
\end{description}

It is possible use one each of the SMCSenseHAT and SMCSenseHAT1 boards to 
handle four separate turnouts.  You should only connect at most one of each of 
these boards on a single Raspberry Pi.



%%%%%%%%%%%%%%%%%%%%%%%%%%%%%%%%%%%%%%%%%%%%%%%%%%%%%%%%%%%%%%%%%%%%%%%%%%%%%
%
%  System        : 
%  Module        : 
%  Object Name   : $RCSfile$
%  Revision      : $Revision$
%  Date          : $Date$
%  Author        : $Author$
%  Created By    : Robert Heller
%  Created       : Wed May 31 19:37:01 2017
%  Last Modified : <170531.2303>
%
%  Description 
%
%  Notes
%
%  History
% 
%%%%%%%%%%%%%%%%%%%%%%%%%%%%%%%%%%%%%%%%%%%%%%%%%%%%%%%%%%%%%%%%%%%%%%%%%%%%%
%
%    Copyright (C) 2017  Robert Heller D/B/A Deepwoods Software
%			51 Locke Hill Road
%			Wendell, MA 01379-9728
%
%    This program is free software; you can redistribute it and/or modify
%    it under the terms of the GNU General Public License as published by
%    the Free Software Foundation; either version 2 of the License, or
%    (at your option) any later version.
%
%    This program is distributed in the hope that it will be useful,
%    but WITHOUT ANY WARRANTY; without even the implied warranty of
%    MERCHANTABILITY or FITNESS FOR A PARTICULAR PURPOSE.  See the
%    GNU General Public License for more details.
%
%    You should have received a copy of the GNU General Public License
%    along with this program; if not, write to the Free Software
%    Foundation, Inc., 675 Mass Ave, Cambridge, MA 02139, USA.
%
% 
%
%%%%%%%%%%%%%%%%%%%%%%%%%%%%%%%%%%%%%%%%%%%%%%%%%%%%%%%%%%%%%%%%%%%%%%%%%%%%%

\documentclass[12pt,notitlepage,twoside]{book}
\usepackage{graphicx}
\usepackage{mathptm}
\usepackage{times}
\usepackage{makeidx}
\usepackage{url}
\pagestyle{headings}
\makeindex
\emergencystretch=50pt
\setcounter{tocdepth}{3}
\setcounter{secnumdepth}{3}
\begin{document}
\title{Railroad Circuits for the Raspberry Pi}
\author{Robert Heller \\ Deepwoods Software \\ Wendell, MA, USA}
\date{\today}
\begin{titlepage}

\maketitle

\clearpage

This documentation was prepared with \LaTeX.

\vspace{.25in}


{\small Copyright \copyright 2017 by Robert Heller D/B/A Deepwoods Software}
\vspace{.25in}

All rights reserved.  Permission is granted to copy this document in
electronic form only, so long as it is with the software it
documents. 

\vspace{.125in}

The author, Robert Heller, may be contacted electronically (E-Mail) via
\url{heller@deepsoft.com}.

\vspace{.25in}

Deepwoods Software's web site URL: \url{http://www.deepsoft.com/}.

\thispagestyle{empty}
\setcounter{page}{0}
\clearpage

\end{titlepage}

\pagenumbering{roman}
\tableofcontents
\listoffigures
\listoftables
\cleardoublepage
\chapter*{Preface}
\addcontentsline{toc}{chapter}{Preface}

This booklet describes a collection of circuit boards I designed to make use 
of Raspberry Pi GPIO pins to be used in the control of a model railroad 
layout.  I am not formally trained as an electronics engineer, but these 
circuits are fairly simple and are readily cribbed from the data sheets of the 
various microchips used.  If someone who is a trained electronics engineer 
finds something wrong, I would of course like to hear about it.

\cleardoublepage
\pagenumbering{arabic}
\chapter{General Information}

All but two of the boards are add on boards for a Raspberry Pi ``B'' model, 
one with a 40-pin header and have the mechanical form factor of a Raspberry Pi 
``HAT''.  Since none of them have an eeprom on them, they are technically not 
actual ``HATs''.  These boards can be ``stacked'', with some limitations, as 
indicated in their respective chapters, mainly because some of these boards 
are hard wired to use specific GPIO pins.  It is therefore possible to use 
``stacking'' (or ``stack through'') headers, headers that are both female 
sockets and male pins.  The end user has a choice of headers to use: standard 
female only headers, long female only headers, standard ``stacking'' headers 
or long/tall ``stacking'' headers.  In the chapters for each board in the 
parts list section, I list the various header options, with part numbers and 
suppliers.

Another place where the end user has choices is in the terminal blocks.  With 
some exceptions (mostly for higher power cases), all of the terminal blocks 
are specified as .1'' (2.54mm) pitch screw terminals.  It is possible to 
substitute other .1'' (2.54mm) pitch connections, including vertical or right 
angle pin arrays or even spring terminals.  This will depend on what sort of 
interconnection technology the end user would prefer.  In the parts list will 
be the various alternative options.

All of the boards use through-hole parts and should be fairly easy to hand 
solder with a basic good quality soldering iron.  All of the components are 
standard and readily available from various suppliers (like Mouser or 
Digi-Key).

All of the boards have mounting holes.  The mounting holes on the HAT-like 
boards align over the mounting holes in the Raspberry Pi.  Using standard 
height headers allows about 3/8'' between boards and using tall headers allows 
1/2'' between boards.  It is probably a good idea to use spacer hardware to 
secure the boards.  Male/female 4-40 threaded hex spacers would quite well.

%%%%%%%%%%%%%%%%%%%%%%%%%%%%%%%%%%%%%%%%%%%%%%%%%%%%%%%%%%%%%%%%%%%%%%%%%%%%%
%
%  System        : 
%  Module        : 
%  Object Name   : $RCSfile$
%  Revision      : $Revision$
%  Date          : $Date$
%  Author        : $Author$
%  Created By    : Robert Heller
%  Created       : Wed May 31 20:05:00 2017
%  Last Modified : <170531.2304>
%
%  Description 
%
%  Notes
%
%  History
% 
%%%%%%%%%%%%%%%%%%%%%%%%%%%%%%%%%%%%%%%%%%%%%%%%%%%%%%%%%%%%%%%%%%%%%%%%%%%%%
%
%    Copyright (C) 2017  Robert Heller D/B/A Deepwoods Software
%			51 Locke Hill Road
%			Wendell, MA 01379-9728
%
%    This program is free software; you can redistribute it and/or modify
%    it under the terms of the GNU General Public License as published by
%    the Free Software Foundation; either version 2 of the License, or
%    (at your option) any later version.
%
%    This program is distributed in the hope that it will be useful,
%    but WITHOUT ANY WARRANTY; without even the implied warranty of
%    MERCHANTABILITY or FITNESS FOR A PARTICULAR PURPOSE.  See the
%    GNU General Public License for more details.
%
%    You should have received a copy of the GNU General Public License
%    along with this program; if not, write to the Free Software
%    Foundation, Inc., 675 Mass Ave, Cambridge, MA 02139, USA.
%
% 
%
%%%%%%%%%%%%%%%%%%%%%%%%%%%%%%%%%%%%%%%%%%%%%%%%%%%%%%%%%%%%%%%%%%%%%%%%%%%%%

\chapter{SMCSenseHAT: Stall Motor Control and Sense HAT}

This is a circuit board to for an add-on board for a Raspberry Pi B+ that will
control  two  stall-motor  turnout  motors for a model  railroad.  It also has
sense  logic to return the state of the  turnouts,  using one pole of the DPDT
contacts in the stall-motor (typical of Tortoise stall-motors).

The circuit board uses a 40pin header socket to connect to the 40pin header on
the  Raspberry Pi B+ and can use a  stack-through  header to allow  additional
boards to be stacked on top of it.

\section{GPIO Pins Used and stacking restrictions.}

This board uses four GPIO pins:

\begin{description}
\item[WiringPi 0, BCM 17] Motor Select 1: select the position of stall motor 
1. 
\item[WiringPi 1, BCM 18] Motor Select 2: select the position of stall motor 
2. 
\item[WiringPi 2, BCM 27] Point Sense 1: return the state of the points for 
stall motor 1. 
\item[WiringPi 3, BCM 22] Point Sense 2: return the state of the points for 
stall motor 2. 
\end{description}

Each of the motor drive circuits is through a transistor that can handle 1 amp 
continuious colector current, which is way more needed to drive typical stall 
motor.  It is enough to drive a pair of stall motors, wired in parallel as 
would be the case for a cross over.

Because this board is hardwired to use four specicic GPIO pins it is not 
possible to use two or more of these boards on given Raspberry Pi.  But in the 
SMCSenseHAT1 directory is a nearly identical board, that uses a different set 
of four GPIO pins:

\begin{description}
\item[WiringPi 4, BCM 23] Motor Select 1: select the position of stall motor 
1. 
\item[WiringPi 5, BCM 24] Motor Select 2: select the position of stall motor 
2. 
\item[WiringPi 6, BCM 25] Point Sense 1: return the state of the points for 
stall motor 1. 
\item[WiringPi 7, BCM 4] Point Sense 2: return the state of the points for 
stall motor 2. 
\end{description}

It is possible use one each of the SMCSenseHAT and SMCSenseHAT1 boards to 
handle four separate turnouts.  You should only connect at most one of each of 
these boards on a single Raspberry Pi.



%%%%%%%%%%%%%%%%%%%%%%%%%%%%%%%%%%%%%%%%%%%%%%%%%%%%%%%%%%%%%%%%%%%%%%%%%%%%%
%
%  System        : 
%  Module        : 
%  Object Name   : $RCSfile$
%  Revision      : $Revision$
%  Date          : $Date$
%  Author        : $Author$
%  Created By    : Robert Heller
%  Created       : Wed May 31 20:07:09 2017
%  Last Modified : <171103.1610>
%
%  Description 
%
%  Notes
%
%  History
% 
%%%%%%%%%%%%%%%%%%%%%%%%%%%%%%%%%%%%%%%%%%%%%%%%%%%%%%%%%%%%%%%%%%%%%%%%%%%%%
%
%    Copyright (C) 2017  Robert Heller D/B/A Deepwoods Software
%			51 Locke Hill Road
%			Wendell, MA 01379-9728
%
%    This program is free software; you can redistribute it and/or modify
%    it under the terms of the GNU General Public License as published by
%    the Free Software Foundation; either version 2 of the License, or
%    (at your option) any later version.
%
%    This program is distributed in the hope that it will be useful,
%    but WITHOUT ANY WARRANTY; without even the implied warranty of
%    MERCHANTABILITY or FITNESS FOR A PARTICULAR PURPOSE.  See the
%    GNU General Public License for more details.
%
%    You should have received a copy of the GNU General Public License
%    along with this program; if not, write to the Free Software
%    Foundation, Inc., 675 Mass Ave, Cambridge, MA 02139, USA.
%
% 
%
%%%%%%%%%%%%%%%%%%%%%%%%%%%%%%%%%%%%%%%%%%%%%%%%%%%%%%%%%%%%%%%%%%%%%%%%%%%%%

\chapter{QuadSSSQuadIn: Quad SSR and Quad 5V Input HAT}

This is a circuit board to for an add-on board for a Raspberry Pi B+ that will
add four 5V logic inputs and four Solid State Relays, using a MCP23008 I2C I/O 
expander.  There is a jumper header to set one of eight addresses for the 
MCP23008 chip.  This allows using more than one of this board or any other 
board featuring a MCP23008 or MCP23016 or MCP23017 chip (up to eight total).

The circuit board uses a 40pin header socket to connect to the 40pin header on
the  Raspberry Pi B+ and can use a  stack-through  header to allow  additional
boards to be stacked on top of it.

\section{Circuit Description}                                                  
 
\begin{figure}[hbpt]\begin{centering}%
\includegraphics[width=5in]{QuadSSSQuadIn.pdf}
\caption{Circuit Diagram of the QuadSSSQuadIn}
\end{centering}\end{figure}
This circuit uses a MCP23008 to expand the Raspberry Pis I/O to 8 additional 
I/O pins.  Four of these pins (0-3) are used to drive a pair of dual 
opto-isolated SSR chip and the remaining 4 (4-7) are driven by a input buffer 
chip that can be driven with 5V logic.  The SSRs can be used to switch 
arbituary trackside devices, since the output side of the SSRs are rated upto 
+/- 400 volts at up to 0.1 Amp (100ma).  The 5V logic inputs are compatible 
with many sensor boards available (partitularly occupancy detector circuits).

\section{Parts List}

\begin{table}[htdp]
\begin{centering}\begin{tabular}{|l|l|p{1in}|l|p{.5in}|}
\hline
Value&Qty&Refs&Mouser Part Number&Adafruit Part Number\\
\hline
.1 uf&3&C1 C2 C3&21RZ310-RC&\\
\hline
ASSR-4128&2&IC1 IC2&630-ASSR-4128-002E&\\
\hline
RPi GPIO&1&J0&855-M20-6102045&2223\\
\hline
CONN 3X2&1&J1&517-929836-02-03&\\
\hline
330 Ohms&1&RP1&652-4608X-AP2-331LF&\\
\hline
10K Ohms&2&RR1 RR2&652-4605X-1LF-10K&\\
\hline
CONN 8&1&T1&651-1725711&\\
\hline
CONN 6&1&T2&651-1725698&\\
\hline
MCP23008&1&U1&579-MCP23008-E/P&\\
\hline
74LV125AN&1&U2&595-SN74LV125AN&\\
\hline
\end{tabular}
\caption{Parts list for QuadSSSQuadIn boards.}
\end{centering}\end{table}\footnote{Mouser Project link: 
\url{http://www.mouser.com/ProjectManager/ProjectDetail.aspx?AccessID=97fe7b85dc}.}

The only parts that might be substituted are J0 (the RPi GPIO Header), and T1
and T2 (the I/O terminals). The parts listed are for the stacking headers for 
the RPi GPIO Header, and screw terminals for the I/O terminals.  Feel free to 
select a non-stacking header for the RPi GPIO Header and to select either pin 
arrays or spring terminals for the T1 and T2.                   

\section{Circuit Board Layout}

\begin{figure}[hbpt]\begin{centering}%
\includegraphics[width=5in]{QuadSSSQuadIn3DTop.png}
\caption{3D rendering of the QuadSSSQuadIn board}
\end{centering}\end{figure}
\begin{figure}[hbpt]\begin{centering}%
\includegraphics[width=5in]{QuadSSSQuadIn.png}
\caption{Fabrication image of the QuadSSSQuadIn board}
\end{centering}\end{figure}
Board assembly is straight forward.  You need to be careful orienting the ICs, 
noting that the SSRs are opositely oriented from the other ICs.  Also the 
SIP resistor arrays need to be carefully oriented -- the dot marks pin 1, 
which is indicated on the board with a square pad\footnote{The first batch of 
the boards I ordered used the wrong PCB modules for the terminals and the 
holes are too small for the screw terminal pins to go all the way in.  They 
can be ``jammed'' in enough to be soldered. Pin arrays fit a little better, 
but still need some effort to seat.  The next batch I order will not have this 
problem.}. 


\include{SignalHAT}
\include{DualUncouplerHAT}
%%%%%%%%%%%%%%%%%%%%%%%%%%%%%%%%%%%%%%%%%%%%%%%%%%%%%%%%%%%%%%%%%%%%%%%%%%%%%
%
%  System        : 
%  Module        : 
%  Object Name   : $RCSfile$
%  Revision      : $Revision$
%  Date          : $Date$
%  Author        : $Author$
%  Created By    : Robert Heller
%  Created       : Wed May 31 20:09:46 2017
%  Last Modified : <171104.0938>
%
%  Description 
%
%  Notes
%
%  History
% 
%%%%%%%%%%%%%%%%%%%%%%%%%%%%%%%%%%%%%%%%%%%%%%%%%%%%%%%%%%%%%%%%%%%%%%%%%%%%%
%
%    Copyright (C) 2017  Robert Heller D/B/A Deepwoods Software
%			51 Locke Hill Road
%			Wendell, MA 01379-9728
%
%    This program is free software; you can redistribute it and/or modify
%    it under the terms of the GNU General Public License as published by
%    the Free Software Foundation; either version 2 of the License, or
%    (at your option) any later version.
%
%    This program is distributed in the hope that it will be useful,
%    but WITHOUT ANY WARRANTY; without even the implied warranty of
%    MERCHANTABILITY or FITNESS FOR A PARTICULAR PURPOSE.  See the
%    GNU General Public License for more details.
%
%    You should have received a copy of the GNU General Public License
%    along with this program; if not, write to the Free Software
%    Foundation, Inc., 675 Mass Ave, Cambridge, MA 02139, USA.
%
% 
%
%%%%%%%%%%%%%%%%%%%%%%%%%%%%%%%%%%%%%%%%%%%%%%%%%%%%%%%%%%%%%%%%%%%%%%%%%%%%%

\chapter{MCP23017Hat: 16 bit GPIO expander HAT}

This is a circuit board to for an add-on board for a Raspberry Pi B+ that will 
add 16 3V GPIO pins, using a MCP23017 I2C I/O expander. There is a jumper 
header to set one of eight addresses for the MCP23017 chip.  This allows using 
more than one of this board or any other board featuring a MCP23008 or 
MCP23016 or MCP23017 chip (up to eight total). 

The circuit board uses a 40pin header socket to connect to the 40pin header on 
the  Raspberry Pi B+ and can use a  stack-through  header to allow  additional 
boards to be stacked on top of it.                                             
 
\section{Circuit Description}          

\begin{figure}[hbpt]\begin{centering}%
\includegraphics[width=5in]{MCP23017.pdf}
\caption{Circuit Diagram of the MCP23017}
\end{centering}\end{figure}
This circuit uses a MCP23017 to expand the Raspberry Pis I/O to 16 additional 
I/O pins. The 16 3V logic pins are simply brought out to screw terminals, 
along with the 3V and ground connections (useful for logic reference).

\section{Parts List}

\begin{table}[htdp]
\begin{centering}\begin{tabular}{|l|l|p{1in}|l|p{.5in}|}
\hline
Value&Qty&Refs&Mouser Part Number&Adafruit Part Number\\
\hline
.1 uf&1&C1&21RZ310-RC&\\
\hline
RPi GPIO&1&J0&855-M20-6102045&2223\\
\hline
Address Jumper&1&J1&517-929836-02-03&\\
\hline
10K Ohms&1&RR1&652-4605X-1LF-10K&\\
\hline
GP0;GP1&2&T1 T2&651-1725737 or 2x 651-1725685&\\
\hline
MCP23017&1&U1&579-MCP23017-E/SP&\\
\hline
\end{tabular}
\caption{Parts list for MCP23017 boards.}
\end{centering}\end{table}\footnote{Mouser Project link: 
\url{http://www.mouser.com/ProjectManager/ProjectDetail.aspx?AccessID=25df06786a}.}

The only parts that might be substituted are J0 (the RPi GPIO Header), and T1
and T2 (the I/O terminals). The parts listed are for the stacking headers for 
the RPi GPIO Header, and screw terminals for the I/O terminals.  Feel free to 
select a non-stacking header for the RPi GPIO Header and to select either pin 
arrays or spring terminals for the T1 and T2.                   

\section{Circuit Board Layout}

\begin{figure}[hbpt]\begin{centering}%
\includegraphics[width=5in]{MCP230173DTop.png}
\caption{3D rendering of the MCP23017 board}
\end{centering}\end{figure}
\begin{figure}[hbpt]\begin{centering}%
\includegraphics[width=5in]{MCP23017.png}
\caption{Fabrication image of the MCP23017 board}
\end{centering}\end{figure}
Board assembly is straight forward. You need to be careful orienting the IC.
Also the SIP resistor array needs to be carefully oriented -- the dot marks
pin 1, which is indicated on the board with a square pad. 

\section{Downloadables and Software Support}

Full design information is available on GitHub here:
\url{https://github.com/RobertPHeller/RPi-RRCircuits/tree/master/MCP23017Hat}.

This board is supported by the Model Railroad System\footnote{Available as a 
free download from Deepwoods Software at this web address: 
\url{http://www.deepsoft.com/home/products/modelrailroadsystem/}.}
\texttt{OpenLCB\_PiMCP23017} and \texttt{OpenLCB\_PiMCP23017\_signal} daemons.
A basic XML file for it is included in its GitHub folder. 



\include{MCP23008Hat}
\include{CANHat}
\include{QuadUncoupler}
\include{QuadSMC}
\cleardoublepage
%\bibliography{MRR}
%\bibliographystyle{plain}
\cleardoublepage
\printindex
\end{document}
